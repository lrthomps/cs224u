% This must be in the first 5 lines to tell arXiv to use pdfLaTeX, which is strongly recommended.
\pdfoutput=1
% In particular, the hyperref package requires pdfLaTeX in order to break URLs across lines.

\documentclass[10pt]{article}

% Remove the "review" option to generate the final version.
\usepackage[]{acl}

% Standard package includes
\usepackage{times}
\usepackage{latexsym}
\usepackage{amsmath}

% For proper rendering and hyphenation of words containing Latin characters (including in bib files)
\usepackage[T1]{fontenc}
% For Vietnamese characters
% \usepackage[T5]{fontenc}
% See https://www.latex-project.org/help/documentation/encguide.pdf for other character sets

% This assumes your files are encoded as UTF8
\usepackage[utf8]{inputenc}

% This is not strictly necessary, and may be commented out,
% but it will improve the layout of the manuscript,
% and will typically save some space.
\usepackage{microtype}
\usepackage[hang,flushmargin]{footmisc}

\title{Title CS224u final paper}

\author{Lara Thompson \\
  Principle Data Scientist @ Salesforce \\
  \texttt{lara.thompson@salesforce.com} \\ \\
  \today
}

\begin{document}
\maketitle
\begin{abstract}
Brief statement %providing context for your ideas, and then a high-level summary of what you did and its significance. This is a required section.
\end{abstract}

\section{Introduction}

% This template uses the ACL 2022 style files. 

% The numbered section headings are merely suggestions. The two un-numbered sections at the end are required, as is a references section. 

% The maximum length of the paper is 8 pages, excluding the two required un-numbered sections, references, and appendices. There is no length limit for these additional sections. Appendices cannot report on core findings of the paper.

% If you have additional questions about requirements for style, formatting, length, etc., please refer to the ACL guidelines: \url{https://acl-org.github.io/ACLPUB/formatting.html}. Unless otherwise specified, we will adopt their requirements.

The consensus is that collaborative filtering will generally outperform content-based recommendation [1]. However, it is only applicable when usage data is available. Collaborative filtering suffers from the cold start problem: new items that have not been consumed before cannot be recommended. Additionally, items that are only of interest to a niche audience are more difficult to recommend because usage data is scarce

\section{Prior Literature}

This section can make extensive use of your lit review prose. \cite{Arora2017}

\section{Data}

Likely to be very detailed if the datasets are new or unfamiliar to the community, or if familiar datasets are being used in new ways.
Includes prior work on them, statistics, and a collection protocol. 

\section{Model}

Flesh out your own approach, perhaps amplifying themes from the `Prior lit' section.

\section{Methods}

The experimental approach, including descriptions of metrics, baseline models, etc. Details about hyperparameters, optimization choices, etc., are probably best given in appendices, unless they are central to the arguments.

Explicitly define the metrics, even the common ones (or at least reference them). Be clear about how the data is split for assessment. 

\section{Results} 

A no-nonsense report of what happened.

\section{Analysis} 

Discussion of what the results mean, what they don’t mean, where they can be improved, etc. These sections vary a lot depending on the nature of the paper. (For papers reporting on experiments with multiple datasets, it can be good to repeats Methods/Results/Analysis in separate (sub)sections for each dataset.)

\section{Conclusion} 

Quickly summarize what the paper did, and then chart out possible future directions that anyone might pursue.

\section*{Known Project Limitations}

For this section, imagine that your reader is a well-intentioned NLP practitioner who is seeking to make use of your data, models, or findings as part of a separate scholarly project, deployed system, or some other kind of real-world intervention. What should such a person know about your work? Especially important here are limitations and biases that might affect this person, their findings, their experiment participants, or the users of their product or service. The idea is that what you say here will be taken into consideration but this well-intentioned user, leading to better outcomes for everyone.

\section*{Authorship Statement}

Explain how the individual authors contributed to the
project. You are free to include whatever information you deem important to convey. For ideas and a model see \url{http://blog.pnas.org/iforc.pdf} (p.~12).
This statement is required even for singly-authored papers, because we want to know whether your project is a collaboration with people outside of the class. Our rationale for this section is that we think this is an important aspect of scholarship in general. Only in extreme cases, and after discussion with the team, would we consider giving separate grades to team members based on this statement.

\bibliography{project}

\appendix

\section{Example Appendix}\label{sec:appendix}

This is an appendix.

\end{document}
